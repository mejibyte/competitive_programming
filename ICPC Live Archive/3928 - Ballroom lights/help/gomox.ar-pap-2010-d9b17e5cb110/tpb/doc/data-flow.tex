\section{10594 - Data Flow}

\textbf{Problema:} 
Dada una red en la que cada enlace tiene asociado un
tiempo de transferencia (por unidad de datos), y todos
los enlaces permiten transferir una cantidad m\'axima
de datos fija $k$, determinar la manera de enviar una cierta
cantidad de datos $m$ desde un nodo $v_A$ a un nodo $v_B$
de manera tal que el tiempo de transferencia sea m\'inimo.

\subsection{Resoluci\'on}

Se modela como un problema de flujo m\'aximo de costo m\'inimo.
Se considera un grafo donde todas las conexiones dadas en la entrada
son ejes de capacidad $k$, y el costo de un eje es el tiempo de
transferencia por unidad.
Para convertir el problema original en un problema de flujo m\'aximo,
se agrega un nodo inicial $v_0$ y un eje $(v_0,v_A)$ con capacidad
$m$ y costo $0$.

El problema de buscar el flujo m\'aximo de costo m\'inimo desde $v_0$ hasta $v_B$
es equivalente al problema original, porque 
si el flujo m\'aximo de $v_0$ a $v_B$ es igual a $m$,
en particular es posible enviar $m$ unidades de $v_A$ a $v_B$.
Inversamente, si el flujo m\'aximo de $v_0$ a $v_B$ es menor que $m$,
no es posible enviar m\'as de $m$ unidades en la red original, ya
que el eje agregado entre $v_0$ a $v_A$ no est\'a saturado, y por
lo tanto el corte m\'inimo se da en ejes de la red original.

El algoritmo utilizado es Ford-Fulkerson, donde en cada paso
el camino de aumento elegido es el de costo m\'inimo. El
camino de aumento de costo m\'inimo se busca con Bellman-Ford.
Como en el grafo original los ejes tienen costos no negativos,
nunca hay ciclos de costo negativo en el grafo residual.

El hecho de que todos los ejes de la red original (antes de
agregar $v_0$) tengan la misma capacidad, permite acotar la
cantidad de pasos de Ford-Fulkerson de la siguiente manera.

La capacidad del eje $(v_0,v_A)$ en el grafo residual
representa el flujo remanente, es decir, la cantidad de
unidades de informaci\'on que todav\'ia no fueron asignadas
a ejes.
Todos los ejes del grafo, exceptuando el eje $(v_0,v_A)$,
tienen la misma capacidad $k$. Como en la red del problema los
ejes no son orientados, cada eje $(v,w)$ tiene asociado
un eje reverso $(w,v)$ de igual capacidad.
Cada eje y su eje reverso pueden encontrarse en $2k + 1$ estados
posibles: o bien ambos tienen asignado flujo $0$, o bien el flujo
es una cantidad de datos $1 \leq c \leq k$ en un sentido,
o bien en el sentido opuesto.

En cada paso de Ford-Fulkerson, si hay un camino de aumento
desde $v_0$ hasta $v_B$, este camino debe ser de la forma
$v_0, v_A, ..., v_B$.
Adem\'as, como no hay ciclos de costo negativo y el camino
es m\'inimo, el camino debe ser simple en nodos (si repitiera
un nodo $v_i$, habr\'ia un ciclo de costo positivo y se
podr\'ia sacar).
Por lo tanto, en un camino de aumento no se vuelve a utilizar
el eje $(v_0,v_A)$ (ni su reverso). Como adem\'as los caminos
de aumento tienen capacidad positiva por definici\'on, la capacidad
residual del eje $(v_0,v_A)$ va disminuyendo a lo largo de las
iteraciones del algoritmo.

Por inducci\'on, se puede ver que mientras el flujo remanente
sea mayor que $k$ (es decir, en todos los pasos excepto en el \'ultimo)
y mientras haya un camino de aumento,
la capacidad del camino de aumento encontrado entre $v_A$ y $v_B$
es $k$, y adem\'as el flujo asignado a cada uno de los restantes ejes
del grafo (exceptuando el eje $(v_0,v_A)$) es o bien $0$ o bien $k$
(es decir, si se pasa flujo por un eje, esto se hace hasta saturar
la capacidad).

En el caso base, el flujo es $0$ y la capacidad del camino de
aumento es $k$. En una iteraci\'on $i > 0$, la capacidad del
camino de aumento entre $v_A$ y $v_B$ est\'a dada por el m\'inimo
de las capacidades de los ejes en el grafo residual. Si un eje
tiene asignado flujo $k$, su capacidad residual es $0$ y la
capacidad residual del eje reverso es $2k$. Si un eje tiene
asignado flujo $0$, su capacidad residual y la de su eje reverso
son ambas $k$.

Despu\'es de la primera iteraci\'on, hay al menos un camino $C_1$
desde $v_A$ hasta $v_B$ cuyos ejes tienen asignado flujo positivo
(esto es porque el algoritmo de Ford-Fulkerson en todo
momento mantiene un conjunto de caminos disjuntos desde $v_0$
hasta $v_B$).
Por lo tanto, en la iteraci\'on $i > 0$, no puede ocurrir que haya
un camino de aumento $C_2$ desde $v_A$ hasta $v_B$ que tenga
capacidad $2k$ (es decir, tomando siempre ejes con flujo
en un sentido y satur\'andolos en el sentido opuesto),
porque si esto ocurriera se podr\'ia construir un ciclo de costo negativo 
combinando los caminos $C_1$ y $C_2$. Esto es as� porque se puede
pasar flujo por $C_2$ en la direcci�n opuesta sin aumentar el costo, 
ya que el costo se estaba pagando antes, pero pasando flujo en la otra
direcci�n, y luego dejar de pasar flujo por $C_1$, lo cual disminuye el costo,
ya que los costos de los ejes son positivos, por lo cual son mayores a $0$.

Por lo tanto, el camino de aumento tiene capacidad $k$.
Una vez determinado esto, los flujos de los ejes se
actualizan restando o sumando $k$ en el eje y en el eje
reverso, de tal manera que cada uno sigue teniendo
correspondientemente flujo $0$ o $k$.

Por \'ultimo, si en alguna iteraci\'on la capacidad residual
de $(v_0,v_A)$ es menor que $k$, el algoritmo termina en esa
iteraci\'on. Dado que todos los caminos de aumento tienen
capacidad $k$, la capacidad residual del eje $(v_0,v_A)$ debe
ser $m\mod k$. Despu\'es de esta \'ultima iteraci\'on, deja de
valer la propiedad sobre el flujo asignado a cada uno de los ejes,
pero la capacidad residual del eje $(v_0,v_A)$ queda en $0$
y por lo tanto el algoritmo termina.

Este razonamiento permite acotar la cantidad de iteraciones
de Ford-Fulkerson por $\lceil\frac{m}{k}\rceil$.
Adem\'as, el costo de encontrar el camino de aumento de costo m\'inimo
con Bellman-Ford es el de hacer a lo sumo $N - 1$ iteraciones
(donde $N$ es la cantidad de nodos del grafo) en las que se
recorren las $M$ aristas del grafo,
representado con listas de adyacencia.
La complejidad del algoritmo resulta entonces de las
$O(\frac{m}{k})$ iteraciones, cada una con un costo de
$O(N M)$, es decir $O(\frac{m}{k} N M)$.

\subsection{Implementaci�n}
\noindent
\ttfamily
\shorthandoff{"}\\
\hlstd{}\hlline{\ 1\ }\\
\hlline{\ 2\ }\hldir{\#include\ $<$iostream$>$}\\
\hlline{\ 3\ }\hlstd{}\hldir{\#include\ $<$map$>$}\\
\hlline{\ 4\ }\hlstd{}\hldir{\#include\ $<$vector$>$}\\
\hlline{\ 5\ }\hlstd{}\hldir{\#include\ $<$list$>$}\\
\hlline{\ 6\ }\hlstd{}\hldir{\#include\ $<$cassert$>$}\\
\hlline{\ 7\ }\hlstd{}\\
\hlline{\ 8\ }\hlkwa{using\ namespace\ }\hlstd{std}\hlsym{;}\\
\hlline{\ 9\ }\hlstd{}\\
\hlline{10\ }\hlkwc{typedef\ }\hlstd{}\hlkwb{long\ long\ int\ }\hlstd{int64}\hlsym{;}\\
\hlline{11\ }\hlstd{}\\
\hlline{12\ }\hldir{\#define\ forsn(i,\ s,\ n)\ for\ (int64\ i\ =\ (s);\ i\ $<$\ (n);\ i++)}\\
\hlline{13\ }\hlstd{}\hldir{\#define\ forn(i,\ n)\ forsn\ (i,\ 0,\ n)}\\
\hlline{14\ }\hlstd{}\hldir{\#define\ forall(x,\ s)\ for\ (typeof((s).begin())\ x\ =\ (s).begin();\ x\ !=\ (s).end();\ x++)}\\
\hlline{15\ }\hlstd{}\\
\hlline{16\ }\hlkwc{typedef\ }\hlstd{vector}\hlsym{$<$}\hlstd{int64}\hlsym{$>$\ }\hlstd{vint}\hlsym{;}\\
\hlline{17\ }\hlstd{}\hlkwc{typedef\ }\hlstd{vector}\hlsym{$<$}\hlstd{vint}\hlsym{$>$\ }\hlstd{vvint}\hlsym{;}\\
\hlline{18\ }\hlstd{\\
\hlline{19\ }int64\ n}\hlsym{;}\\
\hlline{20\ }\hlstd{int64\ capacidad\textunderscore eje0}\hlsym{;}\\
\hlline{21\ }\hlstd{int64\ capacidad\textunderscore enlace}\hlsym{;}\\
\hlline{22\ }\hlstd{vvint\ grafo}\hlsym{;}\hlstd{\ \ }\hlsym{}\hlstd{}\hlslc{//\ listas\ de\ adyacencia}\\
\hlline{23\ }\hlstd{vvint\ costo}\hlsym{;}\hlstd{\ \ }\hlsym{}\hlstd{}\hlslc{//\ matriz}\\
\hlline{24\ }\hlstd{vvint\ flujo}\hlsym{;}\hlstd{\ \ }\hlsym{}\hlstd{}\hlslc{//\ matriz}\\
\hlline{25\ }\hlstd{}\\
\hlline{26\ }\hldir{\#define\ agregar(u,\ v,\ c)\ \{\ $\backslash$}\\
\hlline{27\ }\hldir{\ grafo{[}u{]}.push\textunderscore back(v);\ $\backslash$}\\
\hlline{28\ }\hldir{\ costo{[}u{]}{[}v{]}\ =\ (c);\ $\backslash$}\\
\hlline{29\ }\hldir{\ grafo{[}v{]}.push\textunderscore back(u);\ $\backslash$}\\
\hlline{30\ }\hldir{\ costo{[}v{]}{[}u{]}\ =\ (c);\ $\backslash$}\\
\hlline{31\ }\hldir{\}}\\
\hlline{32\ }\hlstd{}\\
\hlline{33\ }\hldir{\#define\ capacidad(v,\ w)\ (((v)\ ==\ 0\ \textbar \textbar \ (w)\ ==\ 0)\ ?\ capacidad\textunderscore eje0\ :\ capacidad\textunderscore enlace)}\\
\hlline{34\ }\hlstd{}\\
\hlline{35\ }\hlkwb{void\ }\hlstd{}\hlkwd{print\textunderscore grafo\textunderscore residual}\hlstd{}\hlsym{()\ \{}\\
\hlline{36\ }\hlstd{\ cout\ }\hlsym{$<$$<$\ }\hlstd{}\hlstr{"{-}{-}{-}{-}{-}{-}{-}{-}{-}{-}{-}{-}{-}{-}{-}"}\hlstd{\ }\hlsym{$<$$<$\ }\hlstd{endl}\hlsym{;}\\
\hlline{37\ }\hlstd{\ }\hlkwd{forn\ }\hlstd{}\hlsym{(}\hlstd{v}\hlsym{,\ }\hlstd{n}\hlsym{)\ \{}\\
\hlline{38\ }\hlstd{}\hlstd{\ \ }\hlstd{cout\ }\hlsym{$<$$<$\ }\hlstd{v\ }\hlsym{$<$$<$\ }\hlstd{endl}\hlsym{;}\\
\hlline{39\ }\hlstd{}\hlstd{\ \ }\hlstd{}\hlkwd{forall\ }\hlstd{}\hlsym{(}\hlstd{vecino}\hlsym{,\ }\hlstd{grafo}\hlsym{{[}}\hlstd{v}\hlsym{{]})\ \{}\\
\hlline{40\ }\hlstd{}\hlstd{\ \ \ }\hlstd{cout\ }\hlsym{$<$$<$\ }\hlstd{}\hlstr{"{-}{-}$>$\ "}\hlstd{}\hlsym{;}\\
\hlline{41\ }\hlstd{}\hlstd{\ \ \ }\hlstd{cout\ }\hlsym{$<$$<$\ {*}}\hlstd{vecino}\hlsym{;}\\
\hlline{42\ }\hlstd{}\hlstd{\ \ \ }\hlstd{cout\ }\hlsym{$<$$<$\ }\hlstd{}\hlstr{"\ costo\ =\ "}\hlstd{\ }\hlsym{$<$$<$\ }\hlstd{costo}\hlsym{{[}}\hlstd{v}\hlsym{{]}{[}{*}}\hlstd{vecino}\hlsym{{]};}\\
\hlline{43\ }\hlstd{}\hlstd{\ \ \ }\hlstd{cout\ }\hlsym{$<$$<$\ }\hlstd{}\hlstr{"\ flujo\ =\ "}\hlstd{\ }\hlsym{$<$$<$\ }\hlstd{flujo}\hlsym{{[}}\hlstd{v}\hlsym{{]}{[}{*}}\hlstd{vecino}\hlsym{{]}\ $<$$<$\ }\hlstd{}\hlstr{"\ /\ "}\hlstd{\ }\hlsym{$<$$<$\ }\hlstd{}\hlkwd{capacidad}\hlstd{}\hlsym{(}\hlstd{v}\hlsym{,\ {*}}\hlstd{vecino}\hlsym{);}\\
\hlline{44\ }\hlstd{}\hlstd{\ \ \ }\hlstd{cout\ }\hlsym{$<$$<$\ }\hlstd{endl}\hlsym{;}\\
\hlline{45\ }\hlstd{}\hlstd{\ \ }\hlstd{}\hlsym{\}}\\
\hlline{46\ }\hlstd{}\hlstd{\ \ }\hlstd{cout\ }\hlsym{$<$$<$\ }\hlstd{endl}\hlsym{;}\\
\hlline{47\ }\hlstd{\ }\hlsym{\}}\\
\hlline{48\ }\hlstd{}\hlsym{\}}\\
\hlline{49\ }\hlstd{}\\
\hlline{50\ }\hldir{\#define\ INF\ 0x7fffffff}\\
\hlline{51\ }\hlstd{}\hldir{\#define\ No\textunderscore hay\textunderscore camino\ {-}1}\\
\hlline{52\ }\hlstd{}\hldir{\#define\ signo(x)\ ((x)\ $<$\ 0\ ?\ {-}1\ :\ 1)}\\
\hlline{53\ }\hlstd{int64\ }\hlkwd{caumento}\hlstd{}\hlsym{(}\hlstd{vint}\hlsym{\&\ }\hlstd{camino}\hlsym{)\ \{}\\
\hlline{54\ }\hlstd{\ }\hlslc{//\ buscar\ el\ camino\ de\ aumento\ de\ costo\ minimo\ usando\ Bellman{-}Ford}\\
\hlline{55\ }\hlstd{\ vint\ }\hlkwd{dist}\hlstd{}\hlsym{(}\hlstd{n}\hlsym{,\ }\hlstd{INF}\hlsym{);}\\
\hlline{56\ }\hlstd{\ vint\ }\hlkwd{ant}\hlstd{}\hlsym{(}\hlstd{n}\hlsym{,\ {-}}\hlstd{}\hlnum{1}\hlstd{}\hlsym{);}\\
\hlline{57\ }\hlstd{\\
\hlline{58\ }\ dist}\hlsym{{[}}\hlstd{}\hlnum{0}\hlstd{}\hlsym{{]}\ =\ }\hlstd{}\hlnum{0}\hlstd{}\hlsym{;}\\
\hlline{59\ }\hlstd{\\
\hlline{60\ }\ }\hlkwd{forn\ }\hlstd{}\hlsym{(}\hlstd{it}\hlsym{,\ }\hlstd{n\ }\hlsym{{-}\ }\hlstd{}\hlnum{1}\hlstd{}\hlsym{)\ \{}\\
\hlline{61\ }\hlstd{}\hlstd{\ \ }\hlstd{}\hlslc{//\ para\ cada\ arista\ del\ grafo\ residual}\\
\hlline{62\ }\hlstd{}\hlstd{\ \ }\hlstd{}\hlkwb{bool\ }\hlstd{cambio\ }\hlsym{=\ }\hlstd{}\hlkwa{false}\hlstd{}\hlsym{;}\\
\hlline{63\ }\hlstd{}\hlstd{\ \ }\hlstd{}\hlkwd{forn\ }\hlstd{}\hlsym{(}\hlstd{v}\hlsym{,\ }\hlstd{n}\hlsym{)\ }\hlstd{}\hlkwd{forall\ }\hlstd{}\hlsym{(}\hlstd{vecino}\hlsym{,\ }\hlstd{grafo}\hlsym{{[}}\hlstd{v}\hlsym{{]})\ \{}\\
\hlline{64\ }\hlstd{}\hlstd{\ \ \ }\hlstd{}\hlkwb{const\ }\hlstd{int64\ w\ }\hlsym{=\ {*}}\hlstd{vecino}\hlsym{;}\\
\hlline{65\ }\hlstd{}\hlstd{\ \ \ }\hlstd{}\hlkwa{if\ }\hlstd{}\hlsym{(}\hlstd{}\hlkwd{capacidad}\hlstd{}\hlsym{(}\hlstd{v}\hlsym{,\ }\hlstd{w}\hlsym{)\ {-}\ }\hlstd{flujo}\hlsym{{[}}\hlstd{v}\hlsym{{]}{[}}\hlstd{w}\hlsym{{]}\ $<$=\ }\hlstd{}\hlnum{0}\hlstd{}\hlsym{)\ }\hlstd{}\hlkwa{continue}\hlstd{}\hlsym{;}\\
\hlline{66\ }\hlstd{}\hlstd{\ \ \ }\hlstd{}\hlkwa{if\ }\hlstd{}\hlsym{(}\hlstd{dist}\hlsym{{[}}\hlstd{v}\hlsym{{]}\ ==\ }\hlstd{INF}\hlsym{)\ }\hlstd{}\hlkwa{continue}\hlstd{}\hlsym{;}\\
\hlline{67\ }\hlstd{}\hlstd{\ \ \ }\hlstd{int64\ ndist\ }\hlsym{=\ }\hlstd{dist}\hlsym{{[}}\hlstd{v}\hlsym{{]}\ +\ }\hlstd{}\hlkwd{signo}\hlstd{}\hlsym{(}\hlstd{flujo}\hlsym{{[}}\hlstd{v}\hlsym{{]}{[}}\hlstd{w}\hlsym{{]})\ {*}\ }\hlstd{costo}\hlsym{{[}}\hlstd{v}\hlsym{{]}{[}}\hlstd{w}\hlsym{{]};}\\
\hlline{68\ }\hlstd{}\hlstd{\ \ \ }\hlstd{}\hlkwa{if\ }\hlstd{}\hlsym{(}\hlstd{ndist\ }\hlsym{$<$\ }\hlstd{dist}\hlsym{{[}}\hlstd{w}\hlsym{{]})\ \{}\\
\hlline{69\ }\hlstd{}\hlstd{\ \ \ \ }\hlstd{dist}\hlsym{{[}}\hlstd{w}\hlsym{{]}\ =\ }\hlstd{ndist}\hlsym{;}\\
\hlline{70\ }\hlstd{}\hlstd{\ \ \ \ }\hlstd{ant}\hlsym{{[}}\hlstd{w}\hlsym{{]}\ =\ }\hlstd{v}\hlsym{;}\\
\hlline{71\ }\hlstd{}\hlstd{\ \ \ \ }\hlstd{cambio\ }\hlsym{=\ }\hlstd{}\hlkwa{true}\hlstd{}\hlsym{;}\\
\hlline{72\ }\hlstd{}\hlstd{\ \ \ }\hlstd{}\hlsym{\}}\\
\hlline{73\ }\hlstd{}\hlstd{\ \ }\hlstd{}\hlsym{\}}\\
\hlline{74\ }\hlstd{}\hlstd{\ \ }\hlstd{}\hlkwa{if\ }\hlstd{}\hlsym{(!}\hlstd{cambio}\hlsym{)\ }\hlstd{}\hlkwa{break}\hlstd{}\hlsym{;}\\
\hlline{75\ }\hlstd{\ }\hlsym{\}}\\
\hlline{76\ }\hlstd{\\
\hlline{77\ }\ }\hlslc{//\ armar\ el\ camino}\\
\hlline{78\ }\hlstd{\ int64\ costo\textunderscore camino\ }\hlsym{=\ }\hlstd{}\hlnum{0}\hlstd{}\hlsym{;}\\
\hlline{79\ }\hlstd{\ int64\ actual\ }\hlsym{=\ }\hlstd{n\ }\hlsym{{-}\ }\hlstd{}\hlnum{1}\hlstd{}\hlsym{;}\\
\hlline{80\ }\hlstd{\ }\hlkwa{while\ }\hlstd{}\hlsym{(}\hlstd{}\hlkwa{true}\hlstd{}\hlsym{)\ \{}\\
\hlline{81\ }\hlstd{}\hlstd{\ \ }\hlstd{camino}\hlsym{.}\hlstd{}\hlkwd{push\textunderscore back}\hlstd{}\hlsym{(}\hlstd{actual}\hlsym{);}\\
\hlline{82\ }\hlstd{}\hlstd{\ \ }\hlstd{int64\ anterior\ }\hlsym{=\ }\hlstd{ant}\hlsym{{[}}\hlstd{actual}\hlsym{{]};}\\
\hlline{83\ }\hlstd{}\hlstd{\ \ }\hlstd{}\hlkwa{if\ }\hlstd{}\hlsym{(}\hlstd{anterior\ }\hlsym{==\ {-}}\hlstd{}\hlnum{1}\hlstd{}\hlsym{)\ }\hlstd{}\hlkwa{break}\hlstd{}\hlsym{;}\\
\hlline{84\ }\hlstd{}\hlstd{\ \ }\hlstd{costo\textunderscore camino\ }\hlsym{+=\ }\hlstd{}\hlkwd{signo}\hlstd{}\hlsym{(}\hlstd{flujo}\hlsym{{[}}\hlstd{anterior}\hlsym{{]}{[}}\hlstd{actual}\hlsym{{]})\ {*}\ }\hlstd{costo}\hlsym{{[}}\hlstd{anterior}\hlsym{{]}{[}}\hlstd{actual}\hlsym{{]};}\\
\hlline{85\ }\hlstd{}\hlstd{\ \ }\hlstd{actual\ }\hlsym{=\ }\hlstd{anterior}\hlsym{;}\\
\hlline{86\ }\hlstd{\ }\hlsym{\}}\\
\hlline{87\ }\hlstd{\ }\hlkwa{if\ }\hlstd{}\hlsym{(}\hlstd{actual\ }\hlsym{==\ }\hlstd{}\hlnum{0}\hlstd{}\hlsym{)\ \{}\\
\hlline{88\ }\hlstd{}\hlstd{\ \ }\hlstd{}\hlkwa{return\ }\hlstd{costo\textunderscore camino}\hlsym{;}\\
\hlline{89\ }\hlstd{\ }\hlsym{\}\ }\hlstd{}\hlkwa{else\ }\hlstd{}\hlsym{\{}\\
\hlline{90\ }\hlstd{}\hlstd{\ \ }\hlstd{}\hlkwa{return\ }\hlstd{No\textunderscore hay\textunderscore camino}\hlsym{;}\\
\hlline{91\ }\hlstd{\ }\hlsym{\}}\\
\hlline{92\ }\hlstd{}\hlsym{\}}\\
\hlline{93\ }\hlstd{\\
\hlline{94\ }int64\ }\hlkwd{resolver}\hlstd{}\hlsym{()\ \{}\\
\hlline{95\ }\hlstd{\ int64\ flujo\textunderscore total\ }\hlsym{=\ }\hlstd{}\hlnum{0}\hlstd{}\hlsym{;}\\
\hlline{96\ }\hlstd{\ int64\ costo\textunderscore total\ }\hlsym{=\ }\hlstd{}\hlnum{0}\hlstd{}\hlsym{;}\\
\hlline{97\ }\hlstd{\ }\hlkwa{while\ }\hlstd{}\hlsym{(}\hlstd{flujo\textunderscore total\ }\hlsym{$<$\ }\hlstd{capacidad\textunderscore eje0}\hlsym{)\ \{}\\
\hlline{98\ }\hlstd{}\hlstd{\ \ }\hlstd{}\hlslc{//print\textunderscore grafo\textunderscore residual();}\\
\hlline{99\ }\hlstd{}\hlstd{\ \ }\hlstd{vint\ camino\textunderscore aumento}\hlsym{;}\\
\hlline{100\ }\hlstd{}\hlstd{\ \ }\hlstd{int64\ costo\textunderscore camino\ }\hlsym{=\ }\hlstd{}\hlkwd{caumento}\hlstd{}\hlsym{(}\hlstd{camino\textunderscore aumento}\hlsym{);}\\
\hlline{101\ }\hlstd{}\hlstd{\ \ }\hlstd{}\hlkwa{if\ }\hlstd{}\hlsym{(}\hlstd{costo\textunderscore camino\ }\hlsym{==\ }\hlstd{No\textunderscore hay\textunderscore camino}\hlsym{)\ }\hlstd{}\hlkwa{return\ }\hlstd{No\textunderscore hay\textunderscore camino}\hlsym{;}\\
\hlline{102\ }\hlstd{}\hlstd{\ \ }\hlstd{}\hlslc{//\ saturar\ el\ camino\ de\ aumento}\\
\hlline{103\ }\hlstd{}\hlstd{\ \ }\hlstd{int64\ mn\ }\hlsym{=\ }\hlstd{INF}\hlsym{;}\\
\hlline{104\ }\hlstd{}\hlstd{\ \ }\hlstd{\\
\hlline{105\ }}\hlstd{\ \ }\hlstd{}\hlkwd{forn\ }\hlstd{}\hlsym{(}\hlstd{i}\hlsym{,\ }\hlstd{camino\textunderscore aumento}\hlsym{.}\hlstd{}\hlkwd{size}\hlstd{}\hlsym{()\ {-}\ }\hlstd{}\hlnum{1}\hlstd{}\hlsym{)\ \{}\\
\hlline{106\ }\hlstd{}\hlstd{\ \ \ }\hlstd{}\hlkwb{const\ }\hlstd{int64\ v\ }\hlsym{=\ }\hlstd{camino\textunderscore aumento}\hlsym{{[}}\hlstd{i\ }\hlsym{+\ }\hlstd{}\hlnum{1}\hlstd{}\hlsym{{]},\ }\hlstd{w\ }\hlsym{=\ }\hlstd{camino\textunderscore aumento}\hlsym{{[}}\hlstd{i}\hlsym{{]};}\\
\hlline{107\ }\hlstd{}\hlstd{\ \ \ }\hlstd{}\hlkwa{if\ }\hlstd{}\hlsym{(}\hlstd{flujo}\hlsym{{[}}\hlstd{v}\hlsym{{]}{[}}\hlstd{w}\hlsym{{]}\ $<$\ }\hlstd{}\hlnum{0}\hlstd{}\hlsym{)\ \{}\\
\hlline{108\ }\hlstd{}\hlstd{\ \ \ \ }\hlstd{mn\ }\hlsym{=\ }\hlstd{}\hlkwd{min}\hlstd{}\hlsym{(}\hlstd{mn}\hlsym{,\ {-}}\hlstd{flujo}\hlsym{{[}}\hlstd{v}\hlsym{{]}{[}}\hlstd{w}\hlsym{{]});}\\
\hlline{109\ }\hlstd{}\hlstd{\ \ \ }\hlstd{}\hlsym{\}\ }\hlstd{}\hlkwa{else\ }\hlstd{}\hlsym{\{}\\
\hlline{110\ }\hlstd{}\hlstd{\ \ \ \ }\hlstd{mn\ }\hlsym{=\ }\hlstd{}\hlkwd{min}\hlstd{}\hlsym{(}\hlstd{mn}\hlsym{,\ }\hlstd{}\hlkwd{capacidad}\hlstd{}\hlsym{(}\hlstd{v}\hlsym{,\ }\hlstd{w}\hlsym{)\ {-}\ }\hlstd{flujo}\hlsym{{[}}\hlstd{v}\hlsym{{]}{[}}\hlstd{w}\hlsym{{]});}\\
\hlline{111\ }\hlstd{}\hlstd{\ \ \ }\hlstd{}\hlsym{\}}\\
\hlline{112\ }\hlstd{}\hlstd{\ \ }\hlstd{}\hlsym{\}}\\
\hlline{113\ }\hlstd{}\hlstd{\ \ }\hlstd{}\hlkwd{forn\ }\hlstd{}\hlsym{(}\hlstd{i}\hlsym{,\ }\hlstd{camino\textunderscore aumento}\hlsym{.}\hlstd{}\hlkwd{size}\hlstd{}\hlsym{()\ {-}\ }\hlstd{}\hlnum{1}\hlstd{}\hlsym{)\ \{}\\
\hlline{114\ }\hlstd{}\hlstd{\ \ \ }\hlstd{}\hlkwb{const\ }\hlstd{int64\ v\ }\hlsym{=\ }\hlstd{camino\textunderscore aumento}\hlsym{{[}}\hlstd{i\ }\hlsym{+\ }\hlstd{}\hlnum{1}\hlstd{}\hlsym{{]},\ }\hlstd{w\ }\hlsym{=\ }\hlstd{camino\textunderscore aumento}\hlsym{{[}}\hlstd{i}\hlsym{{]};}\\
\hlline{115\ }\hlstd{}\hlstd{\ \ \ }\hlstd{flujo}\hlsym{{[}}\hlstd{v}\hlsym{{]}{[}}\hlstd{w}\hlsym{{]}\ +=\ }\hlstd{mn}\hlsym{;}\\
\hlline{116\ }\hlstd{}\hlstd{\ \ \ }\hlstd{flujo}\hlsym{{[}}\hlstd{w}\hlsym{{]}{[}}\hlstd{v}\hlsym{{]}\ {-}=\ }\hlstd{mn}\hlsym{;}\\
\hlline{117\ }\hlstd{}\hlstd{\ \ }\hlstd{}\hlsym{\}}\\
\hlline{118\ }\hlstd{}\hlstd{\ \ }\hlstd{}\hlkwa{if\ }\hlstd{}\hlsym{(}\hlstd{mn\ }\hlsym{==\ }\hlstd{}\hlnum{0}\hlstd{}\hlsym{)\ }\hlstd{}\hlkwa{return\ }\hlstd{No\textunderscore hay\textunderscore camino}\hlsym{;}\\
\hlline{119\ }\hlstd{}\hlstd{\ \ }\hlstd{flujo\textunderscore total\ }\hlsym{+=\ }\hlstd{mn}\hlsym{;}\\
\hlline{120\ }\hlstd{}\hlstd{\ \ }\hlstd{costo\textunderscore total\ }\hlsym{+=\ }\hlstd{costo\textunderscore camino\ }\hlsym{{*}\ }\hlstd{mn}\hlsym{;}\\
\hlline{121\ }\hlstd{\ }\hlsym{\}}\\
\hlline{122\ }\hlstd{\\
\hlline{123\ }\ }\hlkwa{return\ }\hlstd{costo\textunderscore total}\hlsym{;}\\
\hlline{124\ }\hlstd{}\hlsym{\}}\\
\hlline{125\ }\hlstd{}\\
\hlline{126\ }\hlkwb{int\ }\hlstd{}\hlkwd{main}\hlstd{}\hlsym{()\ \{}\\
\hlline{127\ }\hlstd{\ int64\ m}\hlsym{;}\\
\hlline{128\ }\hlstd{\ }\hlkwa{while\ }\hlstd{}\hlsym{(}\hlstd{cin\ }\hlsym{$>$$>$\ }\hlstd{n\ }\hlsym{$>$$>$\ }\hlstd{m}\hlsym{)\ \{}\\
\hlline{129\ }\hlstd{}\hlstd{\ \ }\hlstd{n}\hlsym{++;}\\
\hlline{130\ }\hlstd{}\hlstd{\ \ }\hlstd{grafo\ }\hlsym{=\ }\hlstd{}\hlkwd{vvint}\hlstd{}\hlsym{(}\hlstd{n}\hlsym{,\ }\hlstd{}\hlkwd{vint}\hlstd{}\hlsym{());}\\
\hlline{131\ }\hlstd{}\hlstd{\ \ }\hlstd{costo\ }\hlsym{=\ }\hlstd{}\hlkwd{vvint}\hlstd{}\hlsym{(}\hlstd{n}\hlsym{,\ }\hlstd{}\hlkwd{vint}\hlstd{}\hlsym{(}\hlstd{n}\hlsym{,\ }\hlstd{}\hlnum{0}\hlstd{}\hlsym{));}\\
\hlline{132\ }\hlstd{}\hlstd{\ \ }\hlstd{flujo\ }\hlsym{=\ }\hlstd{}\hlkwd{vvint}\hlstd{}\hlsym{(}\hlstd{n}\hlsym{,\ }\hlstd{}\hlkwd{vint}\hlstd{}\hlsym{(}\hlstd{n}\hlsym{,\ }\hlstd{}\hlnum{0}\hlstd{}\hlsym{));}\\
\hlline{133\ }\hlstd{\\
\hlline{134\ }}\hlstd{\ \ }\hlstd{}\hlkwd{agregar}\hlstd{}\hlsym{(}\hlstd{}\hlnum{0}\hlstd{}\hlsym{,\ }\hlstd{}\hlnum{1}\hlstd{}\hlsym{,\ }\hlstd{}\hlnum{0}\hlstd{}\hlsym{);}\\
\hlline{135\ }\hlstd{}\hlstd{\ \ }\hlstd{}\hlkwd{forn\ }\hlstd{}\hlsym{(}\hlstd{i}\hlsym{,\ }\hlstd{m}\hlsym{)\ \{}\\
\hlline{136\ }\hlstd{}\hlstd{\ \ \ }\hlstd{int64\ v1}\hlsym{,\ }\hlstd{v2}\hlsym{,\ }\hlstd{c}\hlsym{;}\\
\hlline{137\ }\hlstd{}\hlstd{\ \ \ }\hlstd{cin\ }\hlsym{$>$$>$\ }\hlstd{v1\ }\hlsym{$>$$>$\ }\hlstd{v2\ }\hlsym{$>$$>$\ }\hlstd{c}\hlsym{;}\\
\hlline{138\ }\hlstd{}\hlstd{\ \ \ }\hlstd{}\hlkwd{agregar}\hlstd{}\hlsym{(}\hlstd{v1}\hlsym{,\ }\hlstd{v2}\hlsym{,\ }\hlstd{c}\hlsym{);}\\
\hlline{139\ }\hlstd{}\hlstd{\ \ }\hlstd{}\hlsym{\}}\\
\hlline{140\ }\hlstd{}\hlstd{\ \ }\hlstd{cin\ }\hlsym{$>$$>$\ }\hlstd{capacidad\textunderscore eje0\ }\hlsym{$>$$>$\ }\hlstd{capacidad\textunderscore enlace}\hlsym{;}\\
\hlline{141\ }\hlstd{}\hlstd{\ \ }\hlstd{int64\ r\ }\hlsym{=\ }\hlstd{}\hlkwd{resolver}\hlstd{}\hlsym{();}\\
\hlline{142\ }\hlstd{}\hlstd{\ \ }\hlstd{}\hlkwa{if\ }\hlstd{}\hlsym{(}\hlstd{r\ }\hlsym{==\ }\hlstd{No\textunderscore hay\textunderscore camino}\hlsym{)\ \{}\\
\hlline{143\ }\hlstd{}\hlstd{\ \ \ }\hlstd{cout\ }\hlsym{$<$$<$\ }\hlstd{}\hlstr{"Impossible."}\hlstd{\ }\hlsym{$<$$<$\ }\hlstd{endl}\hlsym{;}\\
\hlline{144\ }\hlstd{}\hlstd{\ \ }\hlstd{}\hlsym{\}\ }\hlstd{}\hlkwa{else\ }\hlstd{}\hlsym{\{}\\
\hlline{145\ }\hlstd{}\hlstd{\ \ \ }\hlstd{cout\ }\hlsym{$<$$<$\ }\hlstd{r\ }\hlsym{$<$$<$\ }\hlstd{endl}\hlsym{;}\\
\hlline{146\ }\hlstd{}\hlstd{\ \ }\hlstd{}\hlsym{\}}\\
\hlline{147\ }\hlstd{\ }\hlsym{\}}\\
\hlline{148\ }\hlstd{\ }\hlkwa{return\ }\hlstd{}\hlnum{0}\hlstd{}\hlsym{;}\\
\hlline{149\ }\hlstd{}\hlsym{\}}\\
\hlline{150\ }\hlstd{}\\
\mbox{}
\normalfont
\shorthandon{"}

