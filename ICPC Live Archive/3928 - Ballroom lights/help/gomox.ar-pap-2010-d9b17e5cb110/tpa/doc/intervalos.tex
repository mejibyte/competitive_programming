\section{10020 - Minimal coverage}

\textbf{Problema:} dado un conjunto de intervalos $[L_i,R_i)$, determinar
un subconjunto que cubra el segmento $[0,M)$ y que sea m\'inimo
(en el sentido de cardinalidad). En el enunciado los intervalos
se notan $[L,R]$, pero como los extremos son enteros y hay que
cubrir intervalos reales, los problemas son equivalentes.

\subsection{Resoluci�n}
Se resolvi\'o mediante una t\'ecnica golosa. La idea del algoritmo
es tomar en cada paso alguno de los intervalos cuyo borde izquierdo
ya se encuentre cubierto (o a lo sumo coincida con el punto m\'as chico
que todav\'ia no est\'e cubierto) y cuyo borde derecho sea lo mayor posible.

\begin{algorithm}[H]
\begin{algorithmic}
\caption{Subconjunto de $A$ de cardinal m\'inimo que cubre a $[0,M)$}
\PARAMS{$A$: conjunto de intervalos $[L,R)$.}
\ENSURE{ $S$: un conjunto m\'inimo de intervalos de $A$ que cubren $[0,M)$.}
\STATE $A^\star :=$ intervalos no vac\'ios de $A$ ordenados crecientemente por $L$.
\STATE $final := 0$
\STATE $S := [\,]$
\WHILE {$final < M$}
  \IF {$A^\star$ es vac\'ia $\lor$ \big($A^\star_0 = [L,R)$ con $L > final$\big)}
    \STATE\COMMENT{No se puede cubrir el intervalo $[0,M)$}
    \STATE Fallar.
  \ELSE
    \STATE $C := \{ [L,R) \gets A^\star \text{ tales que } L \leq final \}$
    \STATE\COMMENT{$C$ corresponde a un prefijo no vac\'io de $A^\star$}
    \STATE $I := [L_I,R_I) \in C$ tal que $R_I = \max\{R : [L,R) \in C\}$
    \IF {$R_I \leq final$}
      \STATE\COMMENT{No se puede cubrir el intervalo $[0,M)$}
      \STATE Fallar.
    \ENDIF
    \STATE Agregar $I$ a $S$.
    \STATE $A^\star := $ sufijo de $A^\star$ omitiendo los primeros $\#C$ elementos
    \STATE $final := R_I$
  \ENDIF
\ENDWHILE
\end{algorithmic}
\end{algorithm}

El algoritmo termina porque el conjunto $C$ nunca es vac\'io, y por lo tanto $A^\star$
se va achicando.
El invariante del algoritmo es que los intervalos en $S$ cubren el
intervalo $[0,final)$.
Al entrar en una iteraci\'on, todav\'ia falta cubrir el intervalo
$[final, M)$. En la rama del {\bf then}, no se puede cubrir
$[final, M)$ y por lo tanto tampoco $[0,M)$.

En la rama del {\bf else}, el conjunto $C$ no es vac\'io.
Corresponde a un prefijo de $A^\star$ porque los elementos de $A^\star$
est\'an ordenados. Si hay soluci\'on, al menos un intervalo de $C$ debe
figurar en ella, porque todos los intervalos restantes de $A^\star$ no
cubren el punto $final$. Se puede encontrar una soluci\'on m\'inima
tomando $I$ como alguno de los intervalos de $C$ cuyo extremo
derecho sea mayor.
Al incluir $I$ en la soluci\'on, todos los dem\'as intervalos
de $C$ quedan cubiertos por el contenido de $S$ y por lo tanto
se pueden excluir de $A^\star$.

Adem\'as, incluir $I$ en la soluci\'on no puede ser una mala
decisi\'on (que ``moleste'' en iteraciones posteriores), porque
el hecho de elegir un valor m\'as grande de $R_I$ hace que
en iteraciones posteriores el valor de $final$ sea mayor y
por lo tanto el conjunto $C$ incluya m\'as intervalos.

Para una entrada de $n$ intervalos, la complejidad del algoritmo
es $O(n \log n)$ en peor caso. El paso m\'as costoso es ordenarlos.
El resto del algoritmo se reduce a
hacer una pasada sobre los intervalos ya ordenados\footnote{
Eventualmente se podr\'ia mejorar la complejidad para que ordenarlos
fuera $O(n)$ haciendo bucket sort. El enunciado afirma que el valor
de los extremos $L_i, R_i$ es a lo sumo $50000$. Pueden ser negativos,
pero para este problema ser\'ia equivalente considerar los
intervalos $[\max(L_i,0),\max(R_i,0))$.
}. Al iterar los intervalos ordenados seg�n su primera componente, no es
necesario construir los $C$ de forma expl\'icita. Simplemente se recorre hasta
que se obtiene el primer intervalo que cumple que $L_i > final$. El m�ximo
$R_i$ que actualiza a $final$ se puede obtener durante la recorrida. Luego
alcanza con seguir iterando desde donde se termin�, ya que los intervalos ya
revisados no son candidatos a tener en cuenta. Como la lista de intervalos se
recorre a lo sumo una sola vez, el ciclo itera $O(n)$ veces,
con lo cual el costo que domina es el de ordenar los intervalos.

Las operaciones para manipular intervalos son $O(1)$ porque
los extremos se representan con enteros de 32 bits.

En la implementaci\'on del algoritmo en C++ no se incluye el {\bf if}
cuya condici\'on es $R_I \leq final$. En ese caso, el
algoritmo falla en la siguiente iteraci\'on, porque si
$A^\star$ no es vac\'ia, el primer intervalo no puede ser
tal que $L \leq final$.

\subsection{Implementaci�n}

\noindent
\ttfamily
\shorthandoff{"}\\
\hlstd{}\hlline{\ 1\ }\hldir{\#include\ $<$stdio.h$>$}\\
\hlline{\ 2\ }\hlstd{}\hldir{\#include\ $<$iostream$>$}\\
\hlline{\ 3\ }\hlstd{}\hldir{\#include\ $<$list$>$}\\
\hlline{\ 4\ }\hlstd{}\\
\hlline{\ 5\ }\hlkwa{using\ namespace\ }\hlstd{std}\hlsym{;}\\
\hlline{\ 6\ }\hlstd{}\\
\hlline{\ 7\ }\hldir{\#define\ foreachin(it,s)\ for\ (typeof(s.begin())\ it\ =\ (s).begin();\ it\ !=\ (s).end();\ it++)}\\
\hlline{\ 8\ }\hlstd{}\\
\hlline{\ 9\ }\hlkwc{typedef\ }\hlstd{pair}\hlsym{$<$}\hlstd{}\hlkwb{int}\hlstd{}\hlsym{,\ }\hlstd{}\hlkwb{int}\hlstd{}\hlsym{$>$\ }\hlstd{Intervalo}\hlsym{;}\\
\hlline{10\ }\hlstd{}\hlkwc{typedef\ }\hlstd{list}\hlsym{$<$}\hlstd{Intervalo}\hlsym{$>$\ }\hlstd{LIntervalo}\hlsym{;}\\
\hlline{11\ }\hlstd{}\\
\hlline{12\ }\hlkwb{bool\ }\hlstd{}\hlkwd{compararPorPrimeraComp}\hlstd{}\hlsym{(}\hlstd{}\hlkwb{const\ }\hlstd{Intervalo}\hlsym{\&\ }\hlstd{p1}\hlsym{,\ }\hlstd{}\hlkwb{const\ }\hlstd{Intervalo}\hlsym{\&\ }\hlstd{p2}\hlsym{)\{}\\
\hlline{13\ }\hlstd{\ }\hlkwa{return\ }\hlstd{p1}\hlsym{.}\hlstd{first\ }\hlsym{$<$\ }\hlstd{p2}\hlsym{.}\hlstd{first}\hlsym{;}\\
\hlline{14\ }\hlstd{}\hlsym{\}}\\
\hlline{15\ }\hlstd{}\\
\hlline{16\ }\hlkwb{bool\ }\hlstd{}\hlkwd{resolver}\hlstd{}\hlsym{(}\hlstd{LIntervalo}\hlsym{\&\ }\hlstd{intervalos}\hlsym{,\ }\hlstd{}\hlkwb{int\ }\hlstd{limite}\hlsym{,\ }\hlstd{LIntervalo}\hlsym{\&\ }\hlstd{solucion}\hlsym{)\ \{}\\
\hlline{17\ }\hlstd{\ intervalos}\hlsym{.}\hlstd{}\hlkwd{sort}\hlstd{}\hlsym{(}\hlstd{compararPorPrimeraComp}\hlsym{);}\\
\hlline{18\ }\hlstd{\ }\hlkwb{int\ }\hlstd{final\ }\hlsym{=\ }\hlstd{}\hlnum{0}\hlstd{}\hlsym{;\ }\hlstd{}\hlslc{//\ mayor\ coordenada\ cubierta\ hasta\ ahora}\\
\hlline{19\ }\hlstd{\ }\hlkwb{int\ }\hlstd{r\textunderscore i\ }\hlsym{=\ }\hlstd{}\hlnum{0}\hlstd{}\hlsym{;}\hlstd{\ \ \ }\hlsym{}\hlstd{}\hlslc{//\ mayor\ coordenada\ hasta\ ahora\ alcanzada}\\
\hlline{20\ }\hlstd{\ Intervalo\ parActual\ }\hlsym{=\ }\hlstd{}\hlkwd{Intervalo}\hlstd{}\hlsym{(}\hlstd{}\hlnum{0}\hlstd{}\hlsym{,\ }\hlstd{}\hlnum{0}\hlstd{}\hlsym{);\ }\hlstd{}\hlslc{//\ posible\ candidato\ a\ ser\ agregado}\\
\hlline{21\ }\hlstd{\\
\hlline{22\ }\ }\hlkwd{foreachin\ }\hlstd{}\hlsym{(}\hlstd{it}\hlsym{,\ }\hlstd{intervalos}\hlsym{)\ \{}\\
\hlline{23\ }\hlstd{}\hlstd{\ \ }\hlstd{}\hlslc{//\ si\ la\ primera\ componente\ esta\ mayor\ que\ final,\ ya\ miramos}\\
\hlline{24\ }\hlstd{}\hlstd{\ \ }\hlstd{}\hlslc{//\ todo\ el\ C,\ entonces\ hay\ que\ agregar\ el\ intervalo}\\
\hlline{25\ }\hlstd{}\hlstd{\ \ }\hlstd{}\hlkwa{if\ }\hlstd{}\hlsym{(}\hlstd{it}\hlsym{{-}$>$}\hlstd{first\ }\hlsym{$>$\ }\hlstd{final}\hlsym{)\ \{}\\
\hlline{26\ }\hlstd{}\hlstd{\ \ \ }\hlstd{solucion}\hlsym{.}\hlstd{}\hlkwd{push\textunderscore back}\hlstd{}\hlsym{(}\hlstd{parActual}\hlsym{);}\\
\hlline{27\ }\hlstd{}\hlstd{\ \ \ }\hlstd{final\ }\hlsym{=\ }\hlstd{r\textunderscore i}\hlsym{;\ }\hlstd{}\hlslc{//\ muevo\ el\ borde\ final}\\
\hlline{28\ }\hlstd{}\hlstd{\ \ }\hlstd{}\hlsym{\}}\\
\hlline{29\ }\hlstd{}\hlstd{\ \ }\hlstd{}\hlslc{//\ si\ la\ primera\ componente\ es\ menor\ que\ final\ miro\ si\ el\ intervalo\ sirve}\\
\hlline{30\ }\hlstd{}\hlstd{\ \ }\hlstd{}\hlkwa{if\ }\hlstd{}\hlsym{(}\hlstd{it}\hlsym{{-}$>$}\hlstd{first\ }\hlsym{$<$=\ }\hlstd{final}\hlsym{)\ \{}\\
\hlline{31\ }\hlstd{}\hlstd{\ \ \ }\hlstd{}\hlslc{//\ sirve\ si\ llega\ mas\ lejos\ que\ lo\ que\ hay\ hasta\ ahora}\\
\hlline{32\ }\hlstd{}\hlstd{\ \ \ }\hlstd{}\hlkwa{if\ }\hlstd{}\hlsym{(}\hlstd{it}\hlsym{{-}$>$}\hlstd{second\ }\hlsym{$>$\ }\hlstd{r\textunderscore i}\hlsym{)\ \{}\\
\hlline{33\ }\hlstd{}\hlstd{\ \ \ \ }\hlstd{r\textunderscore i\ }\hlsym{=\ }\hlstd{it}\hlsym{{-}$>$}\hlstd{second}\hlsym{;}\\
\hlline{34\ }\hlstd{}\hlstd{\ \ \ \ }\hlstd{parActual\ }\hlsym{=\ {*}}\hlstd{it}\hlsym{;}\\
\hlline{35\ }\hlstd{}\hlstd{\ \ \ \ }\hlstd{}\hlslc{//\ si\ ademas\ ya\ llegue\ al\ limite,\ termine}\\
\hlline{36\ }\hlstd{}\hlstd{\ \ \ \ }\hlstd{}\hlkwa{if\ }\hlstd{}\hlsym{(}\hlstd{r\textunderscore i\ }\hlsym{$>$=\ }\hlstd{limite}\hlsym{)\ \{}\\
\hlline{37\ }\hlstd{}\hlstd{\ \ \ \ \ }\hlstd{solucion}\hlsym{.}\hlstd{}\hlkwd{push\textunderscore back}\hlstd{}\hlsym{(}\hlstd{parActual}\hlsym{);}\\
\hlline{38\ }\hlstd{}\hlstd{\ \ \ \ \ }\hlstd{}\hlkwa{break}\hlstd{}\hlsym{;}\\
\hlline{39\ }\hlstd{}\hlstd{\ \ \ \ }\hlstd{}\hlsym{\}\ }\\
\hlline{40\ }\hlstd{}\hlstd{\ \ \ }\hlstd{}\hlsym{\}}\\
\hlline{41\ }\hlstd{}\hlstd{\ \ }\hlstd{}\hlsym{\}\ }\hlstd{}\hlkwa{else\ }\hlstd{}\hlsym{\{}\\
\hlline{42\ }\hlstd{}\hlstd{\ \ \ }\hlstd{}\hlslc{//\ hay\ un\ hueco\ que\ no\ puedo\ cubrir\ con\ ningun\ intervalo}\\
\hlline{43\ }\hlstd{}\hlstd{\ \ \ }\hlstd{}\hlkwa{break}\hlstd{}\hlsym{;}\\
\hlline{44\ }\hlstd{}\hlstd{\ \ }\hlstd{}\hlsym{\}}\\
\hlline{45\ }\hlstd{\ }\hlsym{\}}\\
\hlline{46\ }\hlstd{\\
\hlline{47\ }\ }\hlkwa{return\ }\hlstd{r\textunderscore i\ }\hlsym{$>$=\ }\hlstd{limite}\hlsym{;}\\
\hlline{48\ }\hlstd{}\hlsym{\}}\\
\hlline{49\ }\hlstd{}\\
\hlline{50\ }\hlkwb{int\ }\hlstd{}\hlkwd{main}\hlstd{}\hlsym{(}\hlstd{}\hlkwb{int\ }\hlstd{argc}\hlsym{,\ }\hlstd{}\hlkwb{char\ }\hlstd{}\hlsym{{*}{*}}\hlstd{argv}\hlsym{)}\\
\hlline{51\ }\hlstd{}\hlsym{\{}\\
\hlline{52\ }\hlstd{\ }\hlkwb{int\ }\hlstd{casos}\hlsym{;}\\
\hlline{53\ }\hlstd{\ cin\ }\hlsym{$>$$>$\ }\hlstd{casos}\hlsym{;}\\
\hlline{54\ }\hlstd{\ }\hlkwa{while\ }\hlstd{}\hlsym{(}\hlstd{casos\ }\hlsym{$>$\ }\hlstd{}\hlnum{0}\hlstd{}\hlsym{)\ \{}\\
\hlline{55\ }\hlstd{}\hlstd{\ \ }\hlstd{casos}\hlsym{{-}{-};}\\
\hlline{56\ }\hlstd{}\hlstd{\ \ }\hlstd{}\hlkwb{int\ }\hlstd{limite}\hlsym{,\ }\hlstd{x}\hlsym{,\ }\hlstd{y}\hlsym{;}\\
\hlline{57\ }\hlstd{}\hlstd{\ \ }\hlstd{cin\ }\hlsym{$>$$>$\ }\hlstd{limite}\hlsym{;}\\
\hlline{58\ }\hlstd{}\hlstd{\ \ }\hlstd{x\ }\hlsym{=\ }\hlstd{}\hlnum{1}\hlstd{}\hlsym{;}\\
\hlline{59\ }\hlstd{}\hlstd{\ \ }\hlstd{y\ }\hlsym{=\ }\hlstd{}\hlnum{1}\hlstd{}\hlsym{;}\\
\hlline{60\ }\hlstd{}\hlstd{\ \ }\hlstd{}\hlslc{//\ cargo\ intervalos}\\
\hlline{61\ }\hlstd{}\hlstd{\ \ }\hlstd{LIntervalo\ intervalos}\hlsym{;}\\
\hlline{62\ }\hlstd{}\hlstd{\ \ }\hlstd{}\hlkwa{while\ }\hlstd{}\hlsym{(}\hlstd{y\ }\hlsym{!=\ }\hlstd{}\hlnum{0\ }\hlstd{}\hlsym{\textbar \textbar \ }\hlstd{x\ }\hlsym{!=\ }\hlstd{}\hlnum{0}\hlstd{}\hlsym{)\ \{}\\
\hlline{63\ }\hlstd{}\hlstd{\ \ \ }\hlstd{}\hlkwd{scanf\ }\hlstd{}\hlsym{(}\hlstd{}\hlstr{"\%d\ \%d"}\hlstd{}\hlsym{,\ \&}\hlstd{x}\hlsym{,\ \&}\hlstd{y}\hlsym{);}\\
\hlline{64\ }\hlstd{}\hlstd{\ \ \ }\hlstd{}\hlkwa{if\ }\hlstd{}\hlsym{(}\hlstd{x\ }\hlsym{!=\ }\hlstd{}\hlnum{0\ }\hlstd{}\hlsym{\textbar \textbar \ }\hlstd{y\ }\hlsym{!=\ }\hlstd{}\hlnum{0}\hlstd{}\hlsym{)\ \{}\\
\hlline{65\ }\hlstd{}\hlstd{\ \ \ \ }\hlstd{}\hlkwa{if\ }\hlstd{}\hlsym{(}\hlstd{x\ }\hlsym{$<$\ }\hlstd{y}\hlsym{)\ \{\ }\hlstd{}\hlslc{//para\ no\ cargar\ intervalos\ inutiles}\\
\hlline{66\ }\hlstd{}\hlstd{\ \ \ \ \ }\hlstd{intervalos}\hlsym{.}\hlstd{}\hlkwd{push\textunderscore front}\hlstd{}\hlsym{(}\hlstd{}\hlkwd{Intervalo}\hlstd{}\hlsym{(}\hlstd{x}\hlsym{,}\hlstd{y}\hlsym{));}\\
\hlline{67\ }\hlstd{}\hlstd{\ \ \ \ }\hlstd{}\hlsym{\}}\\
\hlline{68\ }\hlstd{}\hlstd{\ \ \ }\hlstd{}\hlsym{\}}\\
\hlline{69\ }\hlstd{}\hlstd{\ \ }\hlstd{}\hlsym{\}}\\
\hlline{70\ }\hlstd{}\hlstd{\ \ }\hlstd{}\hlslc{//\ resuelvo\ instancia}\\
\hlline{71\ }\hlstd{}\hlstd{\ \ }\hlstd{LIntervalo\ solucion}\hlsym{;}\\
\hlline{72\ }\hlstd{}\hlstd{\ \ }\hlstd{}\hlkwa{if\ }\hlstd{}\hlsym{(}\hlstd{}\hlkwd{resolver}\hlstd{}\hlsym{(}\hlstd{intervalos}\hlsym{,\ }\hlstd{limite}\hlsym{,\ }\hlstd{solucion}\hlsym{))\ \{}\\
\hlline{73\ }\hlstd{}\hlstd{\ \ \ }\hlstd{cout\ }\hlsym{$<$$<$\ }\hlstd{solucion}\hlsym{.}\hlstd{}\hlkwd{size}\hlstd{}\hlsym{()\ $<$$<$\ }\hlstd{endl}\hlsym{;}\\
\hlline{74\ }\hlstd{}\hlstd{\ \ \ }\hlstd{}\hlkwd{foreachin\ }\hlstd{}\hlsym{(}\hlstd{it}\hlsym{,\ }\hlstd{solucion}\hlsym{)\ \{}\\
\hlline{75\ }\hlstd{}\hlstd{\ \ \ \ }\hlstd{cout\ }\hlsym{$<$$<$\ }\hlstd{it}\hlsym{{-}$>$}\hlstd{first\ }\hlsym{$<$$<$\ }\hlstd{}\hlstr{"\ "}\hlstd{\ }\hlsym{$<$$<$\ }\hlstd{it}\hlsym{{-}$>$}\hlstd{second\ }\hlsym{$<$$<$\ }\hlstd{endl}\hlsym{;}\\
\hlline{76\ }\hlstd{}\hlstd{\ \ \ }\hlstd{}\hlsym{\}}\\
\hlline{77\ }\hlstd{}\hlstd{\ \ }\hlstd{}\hlsym{\}\ }\hlstd{}\hlkwa{else\ }\hlstd{}\hlsym{\{}\\
\hlline{78\ }\hlstd{}\hlstd{\ \ \ }\hlstd{cout\ }\hlsym{$<$$<$\ }\hlstd{}\hlnum{0\ }\hlstd{}\hlsym{$<$$<$\ }\hlstd{endl}\hlsym{;}\\
\hlline{79\ }\hlstd{}\hlstd{\ \ }\hlstd{}\hlsym{\}}\\
\hlline{80\ }\hlstd{}\hlstd{\ \ }\hlstd{cout\ }\hlsym{$<$$<$\ }\hlstd{endl}\hlsym{;}\\
\hlline{81\ }\hlstd{\ }\hlsym{\}}\\
\hlline{82\ }\hlstd{}\hlsym{\}}\\
\hlline{83\ }\hlstd{}\\
\mbox{}
\normalfont
\shorthandon{"}


