\documentclass[10pt,a4paper,notitlepage]{article}
\usepackage[spanish]{babel}
\usepackage{amsmath,amssymb}
\usepackage{algorithmic}

\begin{document}

\section{100 - The $3n + 1$ problem}

\section{861 - Little bishops}

\section{10020 - Minimal coverage}

Problema: dado un conjunto de intervalos $[L_i,R_i)$, determinar
un subconjunto que cubra el segmento $[0,M)$ y que sea m\'inimo
(en el sentido de cardinalidad). En el enunciado los intervalos
se notan $[L,R]$, pero como los extremos son enteros y hay que
cubrir intervalos reales, los problemas son equivalentes.

Se resolvi\'o mediante una t\'ecnica golosa. La idea del algoritmo
es tomar en cada paso alguno de los intervalos cuyo borde izquierdo
ya se encuentre cubierto (o a lo sumo coincida con el punto m\'as chico
que todav\'ia no est\'e cubierto) y cuyo borde derecho sea lo mayor posible.

\vspace{10pt}
\begin{algorithmic}
\STATE Entrada: $A$: conjunto de intervalos $[L,R)$.
\STATE Salida: $S$: un conjunto m\'inimo de intervalos de $A$ que cubren $[0,M)$.
\STATE $A^\star :=$ intervalos no vac\'ios de $A$ ordenados crecientemente por $L$.
\STATE $final := 0$
\STATE $S := [\,]$
\WHILE {$final < M$}
  \IF {$A^\star$ es vac\'ia $\lor$ \big($A^\star_0 = [L,R)$ con $L > final$\big)}
    \STATE --- No se puede cubrir el intervalo $[0,M)$ ---
    \STATE Fallar.
  \ELSE
    \STATE $C := \{ [L,R) \gets A^\star \text{ tales que } L \leq final \}$
    \STATE --- $C$ corresponde a un prefijo no vac\'io de $A^\star$ ---
    \STATE $I := [L_I,R_I) \in C$ tal que $R_I = \max\{R : [L,R) \in C\}$
    \IF {$R_I \leq final$}
      \STATE --- No se puede cubrir el intervalo $[0,M)$ ---
      \STATE Fallar.
    \ENDIF
    \STATE Agregar $I$ a $S$.
    \STATE $A^\star := $ sufijo de $A^\star$ omitiendo los primeros $\#C$ elementos
    \STATE $final := R_I$
  \ENDIF
\ENDWHILE
\end{algorithmic}
\vspace{10pt}

El algoritmo termina porque en la rama del {\bf else}
el conjunto $C$ nunca es vac\'io, y por lo tanto $A^\star$
se va achicando.

El invariante del algoritmo es que los intervalos en $S$ cubren el
intervalo $[0,final)$.
Al entrar en una iteraci\'on, todav\'ia falta cubrir el intervalo
$[final, M)$. En la rama del {\bf then}, no se puede cubrir
$[final, M)$ y por lo tanto tampoco $[0,M)$.

En la rama del {\bf else}, es claro que el conjunto $C$ no es vac\'io.
Corresponde a un prefijo de $A^\star$ porque los elementos de $A^\star$
est\'an ordenados. Si hay soluci\'on, al menos un intervalo de $C$ debe
figurar en ella, porque todos los intervalos restantes de $A^\star$ no
cubren el punto $final$. Se puede encontrar una soluci\'on m\'inima
tomando $I$ como alguno de los intervalos de $C$ cuyo extremo
derecho sea mayor.
Al incluir $I$ en la soluci\'on, todos los dem\'as intervalos
de $C$ quedan cubiertos por el contenido de $S$ y por lo tanto
se pueden excluir de $A^\star$.

Adem\'as, incluir $I$ en la soluci\'on no puede ser una mala
decisi\'on (que ``moleste'' en iteraciones posteriores), porque
el hecho de elegir un valor m\'as grande de $R_I$ hace que
en iteraciones posteriores el valor de $final$ sea mayor y
por lo tanto el conjunto $C$ incluya m\'as intervalos.

Para una entrada de $n$ intervalos, la complejidad del algoritmo
es $O(n \log n)$ en peor caso. El paso m\'as costoso es ordenarlos.
Es claro que el resto del algoritmo se reduce a
hacer una pasada sobre los intervalos ya ordenados\footnote{
Se podr\'ia mejorar la complejidad para que ordenarlos
fuera $O(n)$ haciendo bucket sort. El enunciado afirma que el valor
de los extremos $L_i, R_i$ es a lo sumo $50000$. Pueden ser negativos,
pero para este problema ser\'ia equivalente considerar los
intervalos $[\max(L_i,0),\max(R_i,0))$.
}.
Las operaciones para manipular intervalos son $O(1)$ porque
los extremos se representan con enteros de 32 bits.

En la implementaci\'on del algoritmo en C++ no se incluye el {\bf if}
cuya condici\'on es $R_I \leq final$. En ese caso, el
algoritmo falla en la siguiente iteraci\'on, porque si
$A^\star$ no es vac\'ia, el primer intervalo no puede ser
tal que $L \leq final$.

\section{10069 - Distinct subsequences}

Problema: dadas dos strings $x = x_0 \hdots x_n$ y $z = z_0 \hdots z_m$,
determinar la cantidad de ocurrencias distintas de $z$ como subsecuencia de $x$.
(Dos ocurrencias son distintas si les corresponden distintas secuencias
de \'indices).

Se resolvi\'o mediante un algoritmo de programaci\'on din\'amica.
Se define la siguiente funci\'on $M(i,j)$ para $i \in [0,n]$, $j \in [0,m]$:
$$M(i,j) = \text{cantidad de ocurrencias de $z[0..j)$ como subsecuencia de $x[0..i)$}$$

\noindent
Definida concretamente por las siguientes ecuaciones:

\vspace{10pt}
\begin{tabular}{ll}
$M(i,0) = 1$ &\vspace{5pt}\\
$M(0,j) = 0$ \text{ (para $j > 0$)}&\vspace{5pt}\\
$M(i,j) = M(i - 1, j) + \begin{cases}
M(i - 1, j - 1) & \text{ si $x_{i - 1}$ = $z_{j - 1}$ }\\
0 & \text{ si no}
\end{cases}$
& (para $i, j > 0$)
\end{tabular}
\vspace{10pt}

La primera ecuaci\'on afirma que la cadena vac\'ia ocurre una vez en cualquier otra cadena
(la secuencia de \'indices de la cadena vac\'ia es una secuencia vac\'ia).
La segunda ecuaci\'on afirma que cualquier cadena no vac\'ia no ocurre en la cadena vac\'ia.
En la tercera ecuaci\'on se afirma que:
\begin{itemize}
\item Todas las ocurrencias de $z[0..j)$ en $x[0..i-1)$
      son ocurrencias de $z[0..j)$ en $x[0..i)$.
\item Adem\'as, si $x_{i - 1} = z_{j - 1}$, cada una de las ocurrencias de
      $z[0..j-1)$ en $x[0..i-1)$ tiene asociada una ocurrencia de
      $z[0..j)$ en $x[0..i)$.
\item Es claro que no puede haber otras ocurrencias de $z[0..j)$ en $x[0..i)$
      y que los dos tipos de ocurrencias descriptos son disjuntos,
      de modo que alcanza con sumarlos.
\end{itemize}

El algoritmo se implementa computando la matriz de la funci\'on $M$,
visitando una vez cada posici\'on de la matriz. S\'olo se necesita el
resultado, $M(n,m) = $ cantidad de ocurrencias de $z$ en $x$. Alcanza
con mantener en memoria s\'olo las dos \'ultimas columnas de $M(i,j)$.
La complejidad temporal es $O(n \cdot m)$ en peor caso.

\subsection{Aritm\'etica de n\'umeros grandes}

El enunciado afirma que $|x| \leq 10000$, $|z| \leq 100$ y que la cantidad de
ocurrencias de $z$ en $x$ es a lo sumo $10^{100}$, que obviamente no cabe en registros
de 64 bits. Adem\'as, la afirmaci\'on no es espuria porque {\em pueden} darse
situaciones en las que el n\'umero de ocurrencias sea mucho mayor que $10^{100}$.
Por ejemplo, tomando $x = a^{10000}$ y $z = a^{100}$, hay
$10000 \choose 100$ ocurrencias de $z$ en $x$.

La \'unica operaci\'on requerida por el algoritmo es la suma de n\'umeros
grandes. Se program\'o una clase \texttt{BigInt} que representa
un n\'umero natural con un vector de 16 d\'igitos en base
$10^8$. Esto permite representar enteros en $[0,10^{128})$
e imprimir la salida en base 10 sin recurrir a divisiones.

La suma de n\'umeros grandes es $O(1)$ porque est\'an representados
con un arreglo que siempre tiene exactamente 16 d\'igitos. La
implementaci\'on de la suma con carry es la usual.

\end{document}

