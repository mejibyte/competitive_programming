\documentclass[10pt,a4paper,twoside]{article}

%---------------------------------------------------------------
\usepackage[utf8]{inputenc}
\usepackage[spanish]{babel}
\usepackage{listings}
\usepackage{color}
%---------------------------------------------------------------

\begin{document}

%---------------------------------------------------------------
\title{Resumen de algoritmos para torneos de programación}
\author{Andrés Mejía}
\date{\today}
\maketitle
%---------------------------------------------------------------

%---------------------------------------------------------------
\tableofcontents
\lstlistoflistings
\lstloadlanguages{C++}
%---------------------------------------------------------------

%---------------------------------------------------------------
\definecolor{colorkeywords}{rgb}{0.5, 0.0, 0.3}
\definecolor{colorcomments}{rgb}{0.2, 0.4, 0.3}
\definecolor{colorstrings}{rgb}{0.2, 0.0, 1.0}
\lstset{frame=tRBl}
\lstset{keywordstyle=\color{colorkeywords}\bfseries}
\lstset{commentstyle=\color{colorcomments}\textit}
\lstset{stringstyle=\color{colorstrings}}
\lstset{numbers=left, numberstyle=\tiny, stepnumber=2, numbersep=5pt}
%---------------------------------------------------------------

%---------------------------------------------------------------
\section{Teoría de números}
%---------------------------------------------------------------
\subsection{Big mod}
\lstset{language=c++}
%\lstset{backgroundcolor=listinggray,framerulecolor=blue}
%\lstset{backgroundcolor=listinggray,rulecolor=blue}
%\lstset{backgroundcolor=\color{listinggray},rulecolor=\color{blue}}
%\lstset{linewidth=\textwidth}
%\lstset{labelstep=10}
%\lstset{commentstyle=\textit, stringstyle=\upshape,stringspaces=false}
%\lstset{commentstyle=\textit, stringstyle=\upshape,showspaces=false}
\lstinputlisting[caption=Big mod]{./src/number_theory/bigmod.cpp}

\subsection{Criba de Eratóstenes}
Marca los números primos en un arreglo. Algunos tiempos de ejecución:
\centering
\begin{tabular}{c c}
\hline\hline
SIZE & Tiempo (s) \\ [0.5ex]
\hline
100000 & 0.004 \\
1000000 & 0.078 \\
10000000 & 1.550 \\
100000000 & 14.319 \\ [1ex]
\hline
\end{tabular}
\lstinputlisting[caption=Criba de Eratóstenes]{./src/number_theory/criba.cpp}

\subsection{Divisores de un número}
Este algoritmo imprime todos los divisores de un número (en desorden) en O($\sqrt{n}$).
Hasta 4294967295 (máximo \textit{unsigned long}) responde instantaneamente. Se puede
forzar un poco más usando \textit{unsigned long long} pero más allá de $10^{12}$ empieza a
responder muy lento.

\lstinputlisting[caption=Divisores]{./src/number_theory/divisores.cpp}


\section{Programación dinámica}
\subsection{Longest common subsequence}
\lstinputlisting[caption=Longest common subsequence]{./src/dp/lcs.cpp}


\end{document}
%---------------------------------------------------------------
